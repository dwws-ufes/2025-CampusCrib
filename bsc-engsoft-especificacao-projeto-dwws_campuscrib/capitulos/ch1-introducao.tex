% ==============================================================================
% Projeto de Sistema - Gabriel Zonatelle Borges e Kaio Silva Rosa
% Capítulo 1 - Introdução
% ==============================================================================
\chapter{Introdução}
\label{sec-intro}
\vspace{-1cm}

Este documento apresenta o projeto (\textit{design}) do sistema \emph{\imprimirtitulo}. O \emph{\imprimirtitulo} é um sistema para locação de vaga em repúblicas. O usuário poderá buscar por repúblicas com vagas disponíveis próximo à instituição de ensino em que estudará e fazer uma proposta de locação da vaga. Desta forma, o sistema permitirá que usuários - sejam locadores ou locatários - se cadastrem e se autentiquem. Locadores poderão cadastrar e gerenciar repúblicas e suas vagas, definindo preço por vaga, as características do imóvel (número de quartos, banheiros, localização e imagens), políticas de aceitação de \textit{pets} e gêneros permitidos (somente masculino, somente feminino ou ambos). Usuários locatários poderão buscar as repúblicas podendo adicionar filtros como: preço, características das repúblicas, número de vagas disponíveis, aceitação de \textit{pets} e por gênero de aceitação. Ao se interessar por uma república, o usuário locatário poderá fazer uma proposta de locação de vaga (caso haja disponibilidade). O locatário deverá aceitar ou rejeitar a proposta recebida. Usuários menores de idade deverão ter a proposta de locação realizada por um responsável legal. 
A proposta de locação (ou locação) terá um tempo de validade. Usuários locatários poderão avaliar as repúblicas em que moram ou já moraram.

Além desta introdução, este documento está organizado da seguinte forma: 
a Seção~\ref{sec-plataforma} apresenta a plataforma de software utilizada na implementação do sistema;
a Seção~\ref{sec-arquitetura} apresenta a arquitetura de software; por fim, 
a Seção~\ref{sec-frameweb} apresenta os modelos FrameWeb que descrevem os componentes da arquitetura.

